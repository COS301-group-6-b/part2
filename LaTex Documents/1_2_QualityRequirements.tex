\subsubsection{Scalability}
Scalability is going to be second most important quality attribute of the buzz space system. It is thus prioritized as being critical to the system. Scalability can be achieved on the buzz space system on many fronts including load, functional, administrative and generation scalability (Wikipedia, n.d.).\\
\\
Load scalability will be realized through the use dynamic data storage structures which can grow rapidly and also be traversed rapidly during operations such as searching and filtering. Coupled with recursive algorithms (that perform optimally on large data structures), the data storage algorithms (for transferring, searching, encoding/decoding etc.) will increase 10 fold in performance for every 100 users actively making use of the data storage structures.\\
\\
Functional scalability will be realized by continually improving the buzz system features. Features include allowing for the upload of virtually any types of files (if it’s not supported then archiving them before uploads would also work), using new thread pool API’s for threads and implementing methods that limit unnecessary features based on different users bandwidth (while still offering the maximum functionally of the buzz space). Users connected to up-to 1Mbps will receive the least amount of the buzz features, users using a 2-4Mbps line will receive above average features and users with 5Mbps+ lines will receive all of the buzz space features.\\
\\ 
Administrative scalability is not a major concern at the moment because the buzz system is limited to the members & students of the UP Computer Science department, but it will be realized by the Integrability of the buzz system (which is discussed separately in this document).\\
\\
Generation scalability will be realized by using efficient MIME encoding that will complement the communication protocols used to provide swift feedback to multiple users on the system. The encoding will also assist with the transfer and storage of resources uploaded by the users.     
\subsubsection{Security}
Security will be a relatively low level priority (important but not critical) in the Buzz system as more importance will need to be placed on other aspects such as usability and scalability. A Secure system focuses on three main aspects, namely: to detect attacks, resist attacks and recover from attacks.\\
\\
We will, however, in the buzz system only be focusing on resisting attacks as most of the users will be members of an educational institution and we assume that they are ethical and will not try to attack using unconventional means.\\
\\
We will also be using LDAP (Lightweight Directory Access Protocol) to log the users in and have a secure way of connecting (Webopedia.com, 2015). LDAP is a set of protocols based on standards within the x.500 standard (Webopedia.com, 2015)\\
\\
Passwords should also be hashed, salted password hashing is the best way to do this to avoid the use of lookup tables, reverse lookup tables and rainbow lookup tables to find passwords (Crackstation.net, 2014). The SHA256 hashing function with salt added should be sufficient for our needs.\\
\\
The connection to the servers will be done using https and POST requests for purposes of hiding passwords. We will allow for 5 failed attempts to log in to the server and then block that user out for 5 minutes to discourage bot attacks, this might, however, have an impact on usability.\\
\\
\subsubsection{Maintainability}
Maintainability of the buzz space system is considered a low priority, but necessary none the less. Maintainability can be achieved by making use of design patterns that promote modularity within the system (Microsoft Developer Network, n.d.). Defects, errors and updates in the buzz space system will then be easily isolated, allowing effortless repairs/replacements to be done on those modules without effecting the whole buzz system.\\
\\
Backtracking and transmission error & detection algorithms (Auto repeat requests) will maintain the systems efficiently, reliability and safety (Wikipedia, n.d.). This will allow for uncomplicated roll-backs during unavoidable system restores. This ensures probability of approximately 75 percent when it comes to retaining/restoring threads and resources to a specified condition within a given time period.\\
\\
\subsubsection{Testability}
Testability is also approximately leveled with maintainability when it comes to its prioritization within the buzz space system. This quality requirement will be achieved by firstly implementing testing early on in the development phase of the buzz system so that it can naturally extend (via maintainability) with the system as it scales up. The actual tests will be initially run on mock objects against a simple structured solutions (Microsoft Developer Network, n.d.). The modularity of the system will greatly assist the testability of it. The tests conducted on the system include input & output tests to check consistency (of resource transfer for example) and variation, there will also be network tests that monitor data rates and data packet activity to optimize the system’s network performance. Tests will utilize probing and debugging tools to achieve their purpose.\\