\subsubsection{Description}

Architectural Constraints refer to any constraints that the client has placed on the architecture of the system. In our case, our client hasn't provided any concrete constraints yet. Thus I will briefly describe probable constraints. 

\subsubsection{Refernce Framework for system architecture}

It is most likely that we will be using the Java-EE reference framework. There is however a possibility that we will use JavaEE as a reference framework. At the moment the reference framework to be used is still being discussed. We however believe that JavaEE is the way to go. 

\subsubsection{Technologies}

No specific technologies to be used have been provided by the client. It is clear however that the Buzz System will be web based. This automatically indicates that web based technologies will have to be used. This includes HTML5, JavaScript, CSS, etc. See section 1.5 for more information on technologies. 

\subsubsection{Platforms and Devices}

Although not explicitly mentioned by the client a few platform and device constraints are evident from the nature of the Buzz System.

\begin{itemize}
	\item The system should be accessible from the lab computers at the University of Pretoria campus. These computers run both Windows 7 and Arch Linux. Both these operating systems have Mozilla Firefox and Google Chrome installed to browse the web. Thus the Buzz System should be fully accessible from these two browsers. 

	\item The system should also be accessible from mobile devices using mobile browsers like Chrome. Thus the system should be optimized to work on devices of all screen sizes, and also devices with slower Internet connections. 

	\item At the moment Android deployment is out of scope for this project, but it should still be kept in mind that an Android based client will 
\end{itemize}
